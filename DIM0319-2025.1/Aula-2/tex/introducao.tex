\section{Introdução informal para os algoritmos.}

\begin{frame}{Algoritmo?}
  Existem diversas definições dadas à palavra \textbf{algoritmo} (palavra derivada do nome Al-Khwarizmi), entre elas estão:
  \begin{itemize}
    \item Um procedimento passo a passo para a solução de um problema.
    \item Uma sequência detalhada de ações a serem executadas para resolver um problema.
    \item Raciocínio estruturado e finito para solucionar um problema.
  \end{itemize}
\end{frame}

\begin{frame}{Exemplo 01}
  \begin{itemize}
    \item Problema: Encontrar o livro de cálculo dentro do seu quarto.
  \end{itemize}
  \pause
  \begin{enumerate}
    \item Checar sua mochila.
    \item Verificar nos moveis (escrivaninha, guarda-roupa, e etc.).
    \item Olhar debaixo da cama.
    \item Se achou o livro, fique tranquilo.
    \item Se não achou o livro, aceite a derrota.
  \end{enumerate}
\end{frame}

\begin{frame}{Exemplo 02}
  Suponha que você tem acesso a dois recipientes, um com capacidade de 5 litros e outro com capacidade de 3 litros, além disso, você tem total acesso a uma fonte de água!
  \begin{itemize}
    \item Problema: Usando apenas os seus recipientes como é possível obter exatamente 7 litros de água?
    \pause
    \item {\color{red} Desafio}: No mesmo cenário anterior descreva como é possível obter exatamente 4 litros de água.
  \end{itemize}
\end{frame}


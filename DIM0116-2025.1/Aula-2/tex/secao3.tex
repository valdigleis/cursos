\section{Um pouco do ecossistema Java}

\begin{frame}{Quais os componentes do Java}
  \begin{itemize}
    \item JVM (Java Virtual Machine), um computador virtual, que interpreta e executa bytecode Java.
    \item Compilador (javac): Converte código-fonte Java em bytecode.
    \item Ferramentas de desenvolvimento, como javadoc, javap e depuradores.
    \item As bibliotecas padrão, como java.lang, java.util, java.io, etc.
  \end{itemize}
  \pause 
  A critério de curiosidade:
  \begin{itemize}
    \item O JRE  (Java Runtime Environment) é formado por JVM, Bibliotecas padrão e arquivos de configuração.
    \item O JDK  (Java Development Kit) é formado por javac, as Ferramentas de desenvolvimento e o JRE.
  \end{itemize}
\end{frame}

\begin{frame}{Sobre a JVM}
  A JVM é a máquina virtual que executa bytecode, de maneira independente do sistema operacional, nesse sentido o bytecode é o binário da JVM. Além disso, a JVM possui:
  \begin{itemize}
    \item Class Loader: mecanismo para carrega classes na memória.
    \item JIT Compiler (Just-In-Time): Compila partes do código para código nativo em tempo de execução, otimizando o desempenho.
    \item Garbage Collector (GC):  Gerencia a memória automaticamente, liberando objetos não utilizados.
  \end{itemize}
\end{frame}

\begin{frame}{API's e Bibliotecas padrão}
  Java oferece um conjunto robusto de bibliotecas padrão, incluindo:
  \begin{itemize}
    \item Java SE (Standard Edition): Inclui coleções, manipulação de arquivos, threads, etc.
    \item Java EE (chamado Jakarta EE): Frameworks para aplicações corporativas (JPA, Servlets, EJB, etc.).
    \item Java ME (Micro Edition): Para dispositivos embarcados e móveis.
  \end{itemize}
  Sobre API's vale a menção a seguintes:
  \begin{itemize}
    \item Spring: framework para aplicações empresariais, baseada na arquitetura MVC.
    \item Hibernate: framework ORM (Object-Relational Mapping) para banco de dados.
    \item Maven/Gradle: framework para automação e gerenciamento de dependências.
  \end{itemize}
\end{frame}

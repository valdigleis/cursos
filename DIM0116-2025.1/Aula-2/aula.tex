%--------------------------------------------%
% Template Beamer para Apresentações da UFRN %
% by alcemygvseverino@gmail.com              %
% Baseado em MIT Beamer Template			 %
% versao 1.1								 %
% Atualizado em 14/05/2016					 %
%--------------------------------------------%
%\documentclass[handout,t]{beamer}
\documentclass[presentation,t]{beamer}
% Para alterar a linguagem do documento
\usepackage[portuges]{babel}
% Para aceitar caracteres especias deretamente do teclado
\usepackage[utf8]{inputenc}
% Para seguir as normas da ABNT de citacao e referencias
\usepackage[alf]{abntex2cite}
% Para incluir figuras
\usepackage{graphicx}
% Para melhor ajuste da posisao das figuras
\usepackage{float}
% Para ajustar as dimensoes do layout da pagina
\usepackage{geometry}
% Para formatar estrutura e informacoes de formulas matematicas
\usepackage{amsmath}
% Para incluir simbolos especiais em formulas matematicas
\usepackage{amssymb}
% Para incluir links nas referencias
\usepackage{url}
% Para incluir paginas de documentos .pdf externos
\usepackage{pgfpages}
% Para ajustar o estilo dos contadores
\usepackage{enumerate}
% Para modificar a cor do texto
\usepackage{color}
% Para incluir condicoes
\usepackage{ifthen}
% Para colocar legendas em algo que nao e float
\usepackage{capt-of}
% Pacotes para escrever algoritmos em pseudocódigo
\usepackage[portuguese, ruled, linesnumbered]{algorithm2e}
% Para definir o tema do slide
\usetheme{Berlin}
% Para difinir cores e background
\usecolortheme{ufrn}
% Para numerar as figuras
\setbeamertemplate{caption}[numbered]

% Título
\title[DIM0117]{Estruturas de Dados Básicas II}
% Data
\date{\today}
% Autores
\author[Valdigleis]{Valdigleis\inst{1}}
	%\vspace{0.25cm}
	%Autor 02 \inst{2}
%}

% Instituto
\institute[UFRN]{
	\inst{1}%
        Universidade Federal do Rio Grande do Norte\\
        Centro de Ciência Exatas e da Terra\\
        Departamento de Informática e Matemática Aplicada\\
	\url{valdigleis@dimap.ufrn.br}\\
	\vspace{0.25cm}
	%\inst{2}%
	%Departamento\\
}

% Logo do canto inferior direito
\pgfdeclareimage[height=0.7cm]{logo_UFRN}{figuras/logo_UFRN}
\logo{\vspace*{-0.25cm}\pgfuseimage{logo_UFRN}\hspace*{-0.05cm}}


\begin{document}
% Sumário
\frame{\titlepage}
\section[]{}
\begin{frame}{Sumário}
	\tableofcontents
\end{frame}

% seções
\section{Sucessão por recorrência}

\begin{frame}{Sucessão e Recorrência}
  \begin{itemize}
    \item É sabido que diversos problemas em ciências da computação são solucionáveis através do uso de listas ou sequências.
    \item É possível definir uma sequência usando recorrência (recursão)?\pause \ \textbf{SIM}. . .\pause \ {\color{red} como}?
  \end{itemize}
  \pause
  A definição de uma sequência $u: \mathbb{N} \rightarrow A$ usando recorrência consiste em uma tarefa de duas partes, a saber:
  \begin{itemize}
    \item Apresentar explicitamente qual(quais) é(são) o(s) elemento(s) inicial(iniciais) $u_0$ da sequência, ou seja, apresentar o elemento 
    na base da recorrência (condição inicial).\pause
    \item Indicar como se pode obter cada um dos termos da sequência a custa do(s) termo(s) anterior(es).
  \end{itemize}
\end{frame}

\begin{frame}{Exemplos}
  \begin{itemize}
    \item[(a)] Defina a sequência de naturais múltiplos de 5.
    \pause
    \begin{eqnarray*}
      u_0 & = & 0\\ \pause
      u_{n + 1} & = & 5 + u_{n} \ [\forall n \in \mathbb{N}]
    \end{eqnarray*}
    \pause
    \item[(b)] A sequência de Lucas é definida como:
    \pause
    \begin{eqnarray*}
      u_1 & = & 2\\ \pause
      u_2 & = & 1\\ \pause
      u_{n + 1} & = & u_{n} + u_{n - 1} \ [\forall n \in \mathbb{N}, n \geq 2]
    \end{eqnarray*}
  \end{itemize}
\end{frame}

\begin{frame}{Prática}
  \begin{itemize}
    \item {\color{red}Desafio}: Construa a sequência de Collatz usando recorrência.
    \item A sequência inicia com um $n \in \mathbb{N} - \{0\}$, os próximos valores da sequência são obtidos da seguinte forma: se o número 
    atual é par o próximo será sua metade, se o número atual é impar então o próximo é 3 vezes o atual adicionado de 1.
  \end{itemize}
\end{frame}

\begin{frame}{Questões Importantes}
  \begin{itemize}
    \item[(a)] \cite{carmo2013} Sempre que $V$ é um conjunto e $T: V \rightarrow V$ é uma função total. Para cada $a \in V$, existe uma única sequência 
    $u: \mathbb{N} \rightarrow V$ que satisfaz as condições:
    \begin{eqnarray*}
      u_0 & = & a\\
      u_{n + 1} & = & T(u_{n}) \ \ [\forall n \in \mathbb{N}]
    \end{eqnarray*}
    \item[(b)] Qual a problemática da definição de sequências por recorrência?
  \end{itemize}
\end{frame}
\section{Os quatro pilares de OO}

\begin{frame}{Os pilares}
  \begin{itemize}
    \item Abstração: são as classes propriamente ditas, elas são definidas em termos de código e são fundamentais pois define os objetos no sistema.\pause
    \item Encapsulamento: refere-se à prática de restringir o acesso direto aos atributos e métodos internos de um objeto, permitindo o controle desse acesso por meio de métodos específicos.\pause
    \item Herança: permite a criação de novas classes a partir de classes existentes, reutilizando atributos e métodos. Isso promove a reutilização de código e facilita a manutenção.\pause
    \item Polimorfismo: permite que os métodos tenham comportamentos diferentes dependendo do objeto que os invoca ou do contexto em que são utilizados.
  \end{itemize}
\end{frame}

\begin{frame}{Sobre o Encapsulamento}
  O Encapsulamento tem por objetivos:
  \begin{itemize}
    \item Proteção dos dados, ou seja, vita modificações acidentais ou indevidas do estado dos objetos.
    \item Esconder a implementação, o usuário da classe interage apenas com a interface ``pública'', sem precisar conhecer os detalhes internos.
    \item Facilitar a manutenção, isto é, alterações na implementação não devem afetar quem usa a classe.
  \end{itemize}
  \pause
  E quais o níveis de proteção dos dados?\pause O níveis são: privado, protegido e público.
\end{frame}

\begin{frame}{Sobre Herança}
  As heranças podem ser
  \begin{itemize}
    \item Herança Simples: Uma subclasse herda de uma única superclasse. (Java e PHP seguem esse modelo).
    \item Herança Múltipla: Uma classe herda de múltiplas classes (existe em Python, C++, mas não existe em Java).
    \item Herança Multinível (encadeada): Uma classe herda de outra que já herdou de uma terceira.
    \item Herança Hierárquica: Uma superclasse tem várias subclasses.
  \end{itemize}
\end{frame}

\begin{frame}{Sobre Polimorfismo}
  O Polimorfismo tem por objetos:
  \begin{itemize}
    \item Flexibilidade: Permite que um mesmo método ou interface seja usado para diferentes tipos de objetos.
    \item Extensibilidade: Facilita a adição de novos tipos de objetos sem modificar o código existente.
    \item Abstração: Permite trabalhar com objetos de forma genérica, sem precisar conhecer seus tipos específicos.
  \end{itemize}
  \pause
  Como o polimorfismo é implementado?\pause
  \begin{itemize}
    \item Sobrescrita de métodos (override): Uma classe filha redefine um método da classe pai.
    \item Sobrecarga de métodos (overload): vários métodos com o mesmo nome, mas com parâmetros diferentes.
    \item Interfaces e classes abstratas, definem métodos que devem ser implementados pelas classes filhas.
  \end{itemize}
\end{frame}
\section{Um pouco do ecossistema Java}

\begin{frame}{Quais os componentes do Java}
  \begin{itemize}
    \item JVM (Java Virtual Machine), um computador virtual, que interpreta e executa bytecode Java.
    \item Compilador (javac): Converte código-fonte Java em bytecode.
    \item Ferramentas de desenvolvimento, como javadoc, javap e depuradores.
    \item As bibliotecas padrão, como java.lang, java.util, java.io, etc.
  \end{itemize}
  \pause 
  A critério de curiosidade:
  \begin{itemize}
    \item O JRE  (Java Runtime Environment) é formado por JVM, Bibliotecas padrão e arquivos de configuração.
    \item O JDK  (Java Development Kit) é formado por javac, as Ferramentas de desenvolvimento e o JRE.
  \end{itemize}
\end{frame}

\begin{frame}{Sobre a JVM}
  A JVM é a máquina virtual que executa bytecode, de maneira independente do sistema operacional, nesse sentido o bytecode é o binário da JVM. Além disso, a JVM possui:
  \begin{itemize}
    \item Class Loader: mecanismo para carrega classes na memória.
    \item JIT Compiler (Just-In-Time): Compila partes do código para código nativo em tempo de execução, otimizando o desempenho.
    \item Garbage Collector (GC):  Gerencia a memória automaticamente, liberando objetos não utilizados.
  \end{itemize}
\end{frame}

\begin{frame}{API's e Bibliotecas padrão}
  Java oferece um conjunto robusto de bibliotecas padrão, incluindo:
  \begin{itemize}
    \item Java SE (Standard Edition): Inclui coleções, manipulação de arquivos, threads, etc.
    \item Java EE (chamado Jakarta EE): Frameworks para aplicações corporativas (JPA, Servlets, EJB, etc.).
    \item Java ME (Micro Edition): Para dispositivos embarcados e móveis.
  \end{itemize}
  Sobre API's vale a menção a seguintes:
  \begin{itemize}
    \item Spring: framework para aplicações empresariais, baseada na arquitetura MVC.
    \item Hibernate: framework ORM (Object-Relational Mapping) para banco de dados.
    \item Maven/Gradle: framework para automação e gerenciamento de dependências.
  \end{itemize}
\end{frame}


\end{document}
